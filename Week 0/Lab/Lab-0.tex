% Options for packages loaded elsewhere
\PassOptionsToPackage{unicode}{hyperref}
\PassOptionsToPackage{hyphens}{url}
%
\documentclass[
]{article}
\usepackage{amsmath,amssymb}
\usepackage{iftex}
\ifPDFTeX
  \usepackage[T1]{fontenc}
  \usepackage[utf8]{inputenc}
  \usepackage{textcomp} % provide euro and other symbols
\else % if luatex or xetex
  \usepackage{unicode-math} % this also loads fontspec
  \defaultfontfeatures{Scale=MatchLowercase}
  \defaultfontfeatures[\rmfamily]{Ligatures=TeX,Scale=1}
\fi
\usepackage{lmodern}
\ifPDFTeX\else
  % xetex/luatex font selection
\fi
% Use upquote if available, for straight quotes in verbatim environments
\IfFileExists{upquote.sty}{\usepackage{upquote}}{}
\IfFileExists{microtype.sty}{% use microtype if available
  \usepackage[]{microtype}
  \UseMicrotypeSet[protrusion]{basicmath} % disable protrusion for tt fonts
}{}
\makeatletter
\@ifundefined{KOMAClassName}{% if non-KOMA class
  \IfFileExists{parskip.sty}{%
    \usepackage{parskip}
  }{% else
    \setlength{\parindent}{0pt}
    \setlength{\parskip}{6pt plus 2pt minus 1pt}}
}{% if KOMA class
  \KOMAoptions{parskip=half}}
\makeatother
\usepackage{xcolor}
\usepackage[margin=1in]{geometry}
\usepackage{color}
\usepackage{fancyvrb}
\newcommand{\VerbBar}{|}
\newcommand{\VERB}{\Verb[commandchars=\\\{\}]}
\DefineVerbatimEnvironment{Highlighting}{Verbatim}{commandchars=\\\{\}}
% Add ',fontsize=\small' for more characters per line
\usepackage{framed}
\definecolor{shadecolor}{RGB}{248,248,248}
\newenvironment{Shaded}{\begin{snugshade}}{\end{snugshade}}
\newcommand{\AlertTok}[1]{\textcolor[rgb]{0.94,0.16,0.16}{#1}}
\newcommand{\AnnotationTok}[1]{\textcolor[rgb]{0.56,0.35,0.01}{\textbf{\textit{#1}}}}
\newcommand{\AttributeTok}[1]{\textcolor[rgb]{0.13,0.29,0.53}{#1}}
\newcommand{\BaseNTok}[1]{\textcolor[rgb]{0.00,0.00,0.81}{#1}}
\newcommand{\BuiltInTok}[1]{#1}
\newcommand{\CharTok}[1]{\textcolor[rgb]{0.31,0.60,0.02}{#1}}
\newcommand{\CommentTok}[1]{\textcolor[rgb]{0.56,0.35,0.01}{\textit{#1}}}
\newcommand{\CommentVarTok}[1]{\textcolor[rgb]{0.56,0.35,0.01}{\textbf{\textit{#1}}}}
\newcommand{\ConstantTok}[1]{\textcolor[rgb]{0.56,0.35,0.01}{#1}}
\newcommand{\ControlFlowTok}[1]{\textcolor[rgb]{0.13,0.29,0.53}{\textbf{#1}}}
\newcommand{\DataTypeTok}[1]{\textcolor[rgb]{0.13,0.29,0.53}{#1}}
\newcommand{\DecValTok}[1]{\textcolor[rgb]{0.00,0.00,0.81}{#1}}
\newcommand{\DocumentationTok}[1]{\textcolor[rgb]{0.56,0.35,0.01}{\textbf{\textit{#1}}}}
\newcommand{\ErrorTok}[1]{\textcolor[rgb]{0.64,0.00,0.00}{\textbf{#1}}}
\newcommand{\ExtensionTok}[1]{#1}
\newcommand{\FloatTok}[1]{\textcolor[rgb]{0.00,0.00,0.81}{#1}}
\newcommand{\FunctionTok}[1]{\textcolor[rgb]{0.13,0.29,0.53}{\textbf{#1}}}
\newcommand{\ImportTok}[1]{#1}
\newcommand{\InformationTok}[1]{\textcolor[rgb]{0.56,0.35,0.01}{\textbf{\textit{#1}}}}
\newcommand{\KeywordTok}[1]{\textcolor[rgb]{0.13,0.29,0.53}{\textbf{#1}}}
\newcommand{\NormalTok}[1]{#1}
\newcommand{\OperatorTok}[1]{\textcolor[rgb]{0.81,0.36,0.00}{\textbf{#1}}}
\newcommand{\OtherTok}[1]{\textcolor[rgb]{0.56,0.35,0.01}{#1}}
\newcommand{\PreprocessorTok}[1]{\textcolor[rgb]{0.56,0.35,0.01}{\textit{#1}}}
\newcommand{\RegionMarkerTok}[1]{#1}
\newcommand{\SpecialCharTok}[1]{\textcolor[rgb]{0.81,0.36,0.00}{\textbf{#1}}}
\newcommand{\SpecialStringTok}[1]{\textcolor[rgb]{0.31,0.60,0.02}{#1}}
\newcommand{\StringTok}[1]{\textcolor[rgb]{0.31,0.60,0.02}{#1}}
\newcommand{\VariableTok}[1]{\textcolor[rgb]{0.00,0.00,0.00}{#1}}
\newcommand{\VerbatimStringTok}[1]{\textcolor[rgb]{0.31,0.60,0.02}{#1}}
\newcommand{\WarningTok}[1]{\textcolor[rgb]{0.56,0.35,0.01}{\textbf{\textit{#1}}}}
\usepackage{longtable,booktabs,array}
\usepackage{calc} % for calculating minipage widths
% Correct order of tables after \paragraph or \subparagraph
\usepackage{etoolbox}
\makeatletter
\patchcmd\longtable{\par}{\if@noskipsec\mbox{}\fi\par}{}{}
\makeatother
% Allow footnotes in longtable head/foot
\IfFileExists{footnotehyper.sty}{\usepackage{footnotehyper}}{\usepackage{footnote}}
\makesavenoteenv{longtable}
\usepackage{graphicx}
\makeatletter
\def\maxwidth{\ifdim\Gin@nat@width>\linewidth\linewidth\else\Gin@nat@width\fi}
\def\maxheight{\ifdim\Gin@nat@height>\textheight\textheight\else\Gin@nat@height\fi}
\makeatother
% Scale images if necessary, so that they will not overflow the page
% margins by default, and it is still possible to overwrite the defaults
% using explicit options in \includegraphics[width, height, ...]{}
\setkeys{Gin}{width=\maxwidth,height=\maxheight,keepaspectratio}
% Set default figure placement to htbp
\makeatletter
\def\fps@figure{htbp}
\makeatother
\setlength{\emergencystretch}{3em} % prevent overfull lines
\providecommand{\tightlist}{%
  \setlength{\itemsep}{0pt}\setlength{\parskip}{0pt}}
\setcounter{secnumdepth}{-\maxdimen} % remove section numbering
\ifLuaTeX
  \usepackage{selnolig}  % disable illegal ligatures
\fi
\usepackage{bookmark}
\IfFileExists{xurl.sty}{\usepackage{xurl}}{} % add URL line breaks if available
\urlstyle{same}
\hypersetup{
  pdftitle={Lab 0},
  pdfauthor={Ankur Garg, based on work by Haley Tiu, Rachel Hammond, and Nikhil Kalathil},
  hidelinks,
  pdfcreator={LaTeX via pandoc}}

\title{Lab 0}
\author{Ankur Garg, based on work by Haley Tiu, Rachel Hammond, and
Nikhil Kalathil}
\date{01/20/2024}

\begin{document}
\maketitle

\section{Lab 0 - Hello R!}\label{lab-0---hello-r}

Packages are pre-built suites of tools that allow us to use R in the way
that we want. In this lab we will work with three packages:
\texttt{datasauRus} which contains a dataset, \texttt{gapminder} -- a
policy relevant dataset, and \texttt{tidyverse} which is a collection of
packages for doing data analysis in a ``tidy'' way.

Install these packages by running the following in the console. (Note:
you should already have the first three installed.)

\begin{Shaded}
\begin{Highlighting}[]
\FunctionTok{install.packages}\NormalTok{(}\StringTok{"tidyverse"}\NormalTok{)}
\FunctionTok{install.packages}\NormalTok{(}\StringTok{"gapminder"}\NormalTok{)}
\FunctionTok{install.packages}\NormalTok{(}\StringTok{"ggplot2"}\NormalTok{)}
\FunctionTok{install.packages}\NormalTok{(}\StringTok{"datasauRus"}\NormalTok{)}
\end{Highlighting}
\end{Shaded}

Now we need to load the libraries

\begin{Shaded}
\begin{Highlighting}[]
\FunctionTok{library}\NormalTok{(tidyverse) }
\end{Highlighting}
\end{Shaded}

\begin{verbatim}
## -- Attaching core tidyverse packages ------------------------ tidyverse 2.0.0 --
## v dplyr     1.1.4     v readr     2.1.4
## v forcats   1.0.0     v stringr   1.5.1
## v ggplot2   3.5.1     v tibble    3.2.1
## v lubridate 1.9.3     v tidyr     1.3.1
## v purrr     1.0.2     
## -- Conflicts ------------------------------------------ tidyverse_conflicts() --
## x dplyr::filter() masks stats::filter()
## x dplyr::lag()    masks stats::lag()
## i Use the conflicted package (<http://conflicted.r-lib.org/>) to force all conflicts to become errors
\end{verbatim}

\begin{Shaded}
\begin{Highlighting}[]
\FunctionTok{library}\NormalTok{(datasauRus)}
\FunctionTok{library}\NormalTok{(gapminder)}
\FunctionTok{library}\NormalTok{(ggplot2)}
\end{Highlighting}
\end{Shaded}

The data frame we will be working with today is called
\texttt{datasaurus\_dozen} and it's in the \texttt{datasauRus} package.
Actually, this single data frame contains 13 datasets, designed to show
us why data visualization is important and how summary statistics alone
can be misleading. The different datasets are marked by the
\texttt{dataset} variable.

To find out more about the dataset, type the following in your Console:
\texttt{?datasaurus\_dozen}. A question mark before the name of an
object will always bring up its help file. This command must be ran in
the Console.

\begin{enumerate}
\def\labelenumi{\arabic{enumi}.}
\tightlist
\item
  Based on the help file, how many rows and how many columns does the
  \texttt{datasaurus\_dozen} file have? What are the variables included
  in the data frame? Add your responses to your R Markdown file.
\end{enumerate}

ANSWER:

\section{Data visualization and
summary}\label{data-visualization-and-summary}

\begin{enumerate}
\def\labelenumi{\arabic{enumi}.}
\setcounter{enumi}{1}
\tightlist
\item
  Plot \texttt{y} vs.~\texttt{x} for the \texttt{dino} dataset. Start
  with the \texttt{datasaurus\_dozen} and pipe it into the
  \texttt{filter} function to filter for observations where
  \texttt{dataset\ ==\ "dino"}. Store the resulting filtered data frame
  as a new data frame called \texttt{dino\_data}.
\end{enumerate}

\begin{Shaded}
\begin{Highlighting}[]
\NormalTok{datasaurus\_dozen }\SpecialCharTok{|\textgreater{}} 
  \FunctionTok{filter}\NormalTok{(dataset }\SpecialCharTok{==} \StringTok{"dino"}\NormalTok{) }\OtherTok{{-}\textgreater{}}\NormalTok{ dino\_data}
\NormalTok{dino\_data}
\end{Highlighting}
\end{Shaded}

\begin{verbatim}
## # A tibble: 142 x 3
##    dataset     x     y
##    <chr>   <dbl> <dbl>
##  1 dino     55.4  97.2
##  2 dino     51.5  96.0
##  3 dino     46.2  94.5
##  4 dino     42.8  91.4
##  5 dino     40.8  88.3
##  6 dino     38.7  84.9
##  7 dino     35.6  79.9
##  8 dino     33.1  77.6
##  9 dino     29.0  74.5
## 10 dino     26.2  71.4
## # i 132 more rows
\end{verbatim}

\begin{Shaded}
\begin{Highlighting}[]
\FunctionTok{view}\NormalTok{(dino\_data)}
\FunctionTok{head}\NormalTok{(dino\_data)}
\end{Highlighting}
\end{Shaded}

\begin{verbatim}
## # A tibble: 6 x 3
##   dataset     x     y
##   <chr>   <dbl> <dbl>
## 1 dino     55.4  97.2
## 2 dino     51.5  96.0
## 3 dino     46.2  94.5
## 4 dino     42.8  91.4
## 5 dino     40.8  88.3
## 6 dino     38.7  84.9
\end{verbatim}

There is a lot going on here, so let's slow down and unpack it a bit.

First, the pipe operator: \texttt{\%\textgreater{}\%}, takes what comes
before it and sends it as the first argument to what comes after it. So
here, we're saying \texttt{filter} the \texttt{datasaurus\_dozen} data
frame for observations where \texttt{dataset\ ==\ "dino"}.

Second, the assignment operator: \texttt{\textless{}-}, assigns the name
\texttt{dino\_data} to the filtered data frame.

You should now see \texttt{dino\_data} in your environment on the right.
Click on it to look at the data.

Next, we need to visualize these data. We will use the \texttt{ggplot}
function for this. Its first argument is the data you're visualizing.
Next we define the \texttt{aes}thetic mappings. In other words, the
columns of the data that get mapped to certain aesthetic features of the
plot, e.g.~the \texttt{x} axis will represent the variable called
\texttt{x} and the \texttt{y} axis will represent the variable called
\texttt{y}. Then, we add another layer to this plot where we define
which \texttt{geom}etric shapes we want to use to represent each
observation in the data. In this case we want these to be points, hence
\texttt{geom\_point}.

\begin{Shaded}
\begin{Highlighting}[]
\NormalTok{dino\_data }\SpecialCharTok{|\textgreater{}} 
  \FunctionTok{ggplot}\NormalTok{(}\AttributeTok{mapping =} \FunctionTok{aes}\NormalTok{(}\AttributeTok{x =}\NormalTok{ x,}\AttributeTok{y =}\NormalTok{ y))}\SpecialCharTok{+}
  \FunctionTok{geom\_point}\NormalTok{(}\AttributeTok{colour =} \StringTok{"purple"}\NormalTok{)}\SpecialCharTok{+}
  \FunctionTok{labs}\NormalTok{(}\AttributeTok{title =} \StringTok{"Dinodata"}\NormalTok{)}
\end{Highlighting}
\end{Shaded}

\includegraphics{Lab-0_files/figure-latex/unnamed-chunk-4-1.pdf}

If this seems like a lot, it is. And you will learn about the philosophy
of building data visualizations in layer in detail next week. For now,
follow along with the code that is provided.

\begin{enumerate}
\def\labelenumi{\arabic{enumi}.}
\setcounter{enumi}{1}
\tightlist
\item
  Plot \texttt{y} vs.~\texttt{x} for the \texttt{star} dataset. You can
  (and should) reuse code we introduced above, just replace the dataset
  name with the desired dataset.
\end{enumerate}

\begin{Shaded}
\begin{Highlighting}[]
\DocumentationTok{\#\#\# put your new star plot here}
\NormalTok{datasaurus\_dozen }\SpecialCharTok{|\textgreater{}} 
  \FunctionTok{filter}\NormalTok{(dataset}\SpecialCharTok{==}\StringTok{"star"}\NormalTok{) }\OtherTok{{-}\textgreater{}}\NormalTok{star\_data}
\FunctionTok{head}\NormalTok{(star\_data)}
\end{Highlighting}
\end{Shaded}

\begin{verbatim}
## # A tibble: 6 x 3
##   dataset     x     y
##   <chr>   <dbl> <dbl>
## 1 star     58.2  91.9
## 2 star     58.2  92.2
## 3 star     58.7  90.3
## 4 star     57.3  89.9
## 5 star     58.1  92.0
## 6 star     57.5  88.1
\end{verbatim}

\begin{Shaded}
\begin{Highlighting}[]
\NormalTok{star\_data }\SpecialCharTok{|\textgreater{}} 
  \FunctionTok{ggplot}\NormalTok{(}\AttributeTok{mapping =} \FunctionTok{aes}\NormalTok{(}\AttributeTok{x=}\NormalTok{x, }\AttributeTok{y=}\NormalTok{y))}\SpecialCharTok{+}
  \FunctionTok{geom\_point}\NormalTok{(}\AttributeTok{colour=}\StringTok{"purple"}\NormalTok{)}\SpecialCharTok{+}
  \FunctionTok{labs}\NormalTok{(}\AttributeTok{title =} \StringTok{"Star Data"}\NormalTok{)}
\end{Highlighting}
\end{Shaded}

\includegraphics{Lab-0_files/figure-latex/unnamed-chunk-5-1.pdf}

\begin{enumerate}
\def\labelenumi{\arabic{enumi}.}
\setcounter{enumi}{2}
\tightlist
\item
  Plot \texttt{y} vs.~\texttt{x} for the \texttt{circle} dataset. You
  can (and should) reuse code we introduced above, just replace the
  dataset name with the desired dataset.
\end{enumerate}

\begin{Shaded}
\begin{Highlighting}[]
\DocumentationTok{\#\#\# put your new circle plot here}
\NormalTok{datasaurus\_dozen }\SpecialCharTok{|\textgreater{}} 
  \FunctionTok{filter}\NormalTok{(dataset }\SpecialCharTok{==}\StringTok{"circle"}\NormalTok{) }\OtherTok{{-}\textgreater{}}\NormalTok{ circle\_data}
\FunctionTok{head}\NormalTok{(circle\_data)}
\end{Highlighting}
\end{Shaded}

\begin{verbatim}
## # A tibble: 6 x 3
##   dataset     x     y
##   <chr>   <dbl> <dbl>
## 1 circle   56.0  79.3
## 2 circle   50.0  79.0
## 3 circle   51.3  82.4
## 4 circle   51.2  79.2
## 5 circle   44.4  78.2
## 6 circle   45.0  77.9
\end{verbatim}

\begin{Shaded}
\begin{Highlighting}[]
\NormalTok{circle\_data }\SpecialCharTok{|\textgreater{}} 
  \FunctionTok{ggplot}\NormalTok{(}\AttributeTok{mapping =} \FunctionTok{aes}\NormalTok{(}\AttributeTok{x=}\NormalTok{x, }\AttributeTok{y=}\NormalTok{y))}\SpecialCharTok{+}
  \FunctionTok{geom\_point}\NormalTok{(}\AttributeTok{colour=}\StringTok{"purple"}\NormalTok{)}\SpecialCharTok{+}
  \FunctionTok{labs}\NormalTok{(}\AttributeTok{title =} \StringTok{"Circle Data"}\NormalTok{)}
\end{Highlighting}
\end{Shaded}

\includegraphics{Lab-0_files/figure-latex/unnamed-chunk-6-1.pdf}

\begin{enumerate}
\def\labelenumi{\arabic{enumi}.}
\setcounter{enumi}{3}
\tightlist
\item
  Finally, let's plot all datasets at once. In order to do this we will
  make use of facetting.
\end{enumerate}

\begin{Shaded}
\begin{Highlighting}[]
\NormalTok{datasaurus\_dozen }\SpecialCharTok{|\textgreater{}} 
  \FunctionTok{ggplot}\NormalTok{(}\AttributeTok{mapping =} \FunctionTok{aes}\NormalTok{(}\AttributeTok{x =}\NormalTok{ x, }\AttributeTok{y =}\NormalTok{ y, }\AttributeTok{colour =}\NormalTok{ dataset))}\SpecialCharTok{+}
  \FunctionTok{geom\_point}\NormalTok{()}\SpecialCharTok{+}
  \FunctionTok{facet\_wrap}\NormalTok{(}\SpecialCharTok{\textasciitilde{}}\NormalTok{ dataset)}\SpecialCharTok{+}
  \FunctionTok{theme}\NormalTok{(}\AttributeTok{legend.position =} \StringTok{"none"}\NormalTok{)}\SpecialCharTok{+}
  \FunctionTok{labs}\NormalTok{(}\AttributeTok{title =} \StringTok{"Datasaurus Data"}\NormalTok{)}
\end{Highlighting}
\end{Shaded}

\includegraphics{Lab-0_files/figure-latex/unnamed-chunk-7-1.pdf}

\section{Gapminder and policy data}\label{gapminder-and-policy-data}

For the second part of this exercises, we will turn to the Gapminder
dataset, which we will look at in more detail next week. The
\texttt{gapminder} dataset is built into the \texttt{gapminder} package
and has 6 variables with 1704 observations. To view the data, enter
\texttt{view(gapmider)} in your console.

\subsection{Summary Statistics and Data
Calls}\label{summary-statistics-and-data-calls}

We can use \texttt{summary()} to get a simple summary of the variables
in our dataset.

\begin{Shaded}
\begin{Highlighting}[]
\FunctionTok{summary}\NormalTok{(gapminder)}
\end{Highlighting}
\end{Shaded}

\begin{verbatim}
##         country        continent        year         lifeExp     
##  Afghanistan:  12   Africa  :624   Min.   :1952   Min.   :23.60  
##  Albania    :  12   Americas:300   1st Qu.:1966   1st Qu.:48.20  
##  Algeria    :  12   Asia    :396   Median :1980   Median :60.71  
##  Angola     :  12   Europe  :360   Mean   :1980   Mean   :59.47  
##  Argentina  :  12   Oceania : 24   3rd Qu.:1993   3rd Qu.:70.85  
##  Australia  :  12                  Max.   :2007   Max.   :82.60  
##  (Other)    :1632                                                
##       pop              gdpPercap       
##  Min.   :6.001e+04   Min.   :   241.2  
##  1st Qu.:2.794e+06   1st Qu.:  1202.1  
##  Median :7.024e+06   Median :  3531.8  
##  Mean   :2.960e+07   Mean   :  7215.3  
##  3rd Qu.:1.959e+07   3rd Qu.:  9325.5  
##  Max.   :1.319e+09   Max.   :113523.1  
## 
\end{verbatim}

There is also a package called \texttt{skimr} package that we can
download for more robust summary statistics. Lets call it here.

\begin{Shaded}
\begin{Highlighting}[]
\FunctionTok{install.packages}\NormalTok{(}\StringTok{"skimr"}\NormalTok{)}
\end{Highlighting}
\end{Shaded}

And after we install these packages, we can run:

\begin{Shaded}
\begin{Highlighting}[]
\FunctionTok{library}\NormalTok{(skimr)}

\NormalTok{gapminder }\SpecialCharTok{|\textgreater{}}  
  \FunctionTok{skim}\NormalTok{()}
\end{Highlighting}
\end{Shaded}

\begin{longtable}[]{@{}ll@{}}
\caption{Data summary}\tabularnewline
\toprule\noalign{}
\endfirsthead
\endhead
\bottomrule\noalign{}
\endlastfoot
Name & gapminder \\
Number of rows & 1704 \\
Number of columns & 6 \\
\_\_\_\_\_\_\_\_\_\_\_\_\_\_\_\_\_\_\_\_\_\_\_ & \\
Column type frequency: & \\
factor & 2 \\
numeric & 4 \\
\_\_\_\_\_\_\_\_\_\_\_\_\_\_\_\_\_\_\_\_\_\_\_\_ & \\
Group variables & None \\
\end{longtable}

\textbf{Variable type: factor}

\begin{longtable}[]{@{}
  >{\raggedright\arraybackslash}p{(\columnwidth - 10\tabcolsep) * \real{0.1489}}
  >{\raggedleft\arraybackslash}p{(\columnwidth - 10\tabcolsep) * \real{0.1064}}
  >{\raggedleft\arraybackslash}p{(\columnwidth - 10\tabcolsep) * \real{0.1489}}
  >{\raggedright\arraybackslash}p{(\columnwidth - 10\tabcolsep) * \real{0.0851}}
  >{\raggedleft\arraybackslash}p{(\columnwidth - 10\tabcolsep) * \real{0.0957}}
  >{\raggedright\arraybackslash}p{(\columnwidth - 10\tabcolsep) * \real{0.4149}}@{}}
\toprule\noalign{}
\begin{minipage}[b]{\linewidth}\raggedright
skim\_variable
\end{minipage} & \begin{minipage}[b]{\linewidth}\raggedleft
n\_missing
\end{minipage} & \begin{minipage}[b]{\linewidth}\raggedleft
complete\_rate
\end{minipage} & \begin{minipage}[b]{\linewidth}\raggedright
ordered
\end{minipage} & \begin{minipage}[b]{\linewidth}\raggedleft
n\_unique
\end{minipage} & \begin{minipage}[b]{\linewidth}\raggedright
top\_counts
\end{minipage} \\
\midrule\noalign{}
\endhead
\bottomrule\noalign{}
\endlastfoot
country & 0 & 1 & FALSE & 142 & Afg: 12, Alb: 12, Alg: 12, Ang: 12 \\
continent & 0 & 1 & FALSE & 5 & Afr: 624, Asi: 396, Eur: 360, Ame:
300 \\
\end{longtable}

\textbf{Variable type: numeric}

\begin{longtable}[]{@{}
  >{\raggedright\arraybackslash}p{(\columnwidth - 20\tabcolsep) * \real{0.1120}}
  >{\raggedleft\arraybackslash}p{(\columnwidth - 20\tabcolsep) * \real{0.0800}}
  >{\raggedleft\arraybackslash}p{(\columnwidth - 20\tabcolsep) * \real{0.1120}}
  >{\raggedleft\arraybackslash}p{(\columnwidth - 20\tabcolsep) * \real{0.0960}}
  >{\raggedleft\arraybackslash}p{(\columnwidth - 20\tabcolsep) * \real{0.1040}}
  >{\raggedleft\arraybackslash}p{(\columnwidth - 20\tabcolsep) * \real{0.0720}}
  >{\raggedleft\arraybackslash}p{(\columnwidth - 20\tabcolsep) * \real{0.0880}}
  >{\raggedleft\arraybackslash}p{(\columnwidth - 20\tabcolsep) * \real{0.0880}}
  >{\raggedleft\arraybackslash}p{(\columnwidth - 20\tabcolsep) * \real{0.0960}}
  >{\raggedleft\arraybackslash}p{(\columnwidth - 20\tabcolsep) * \real{0.1040}}
  >{\raggedright\arraybackslash}p{(\columnwidth - 20\tabcolsep) * \real{0.0480}}@{}}
\toprule\noalign{}
\begin{minipage}[b]{\linewidth}\raggedright
skim\_variable
\end{minipage} & \begin{minipage}[b]{\linewidth}\raggedleft
n\_missing
\end{minipage} & \begin{minipage}[b]{\linewidth}\raggedleft
complete\_rate
\end{minipage} & \begin{minipage}[b]{\linewidth}\raggedleft
mean
\end{minipage} & \begin{minipage}[b]{\linewidth}\raggedleft
sd
\end{minipage} & \begin{minipage}[b]{\linewidth}\raggedleft
p0
\end{minipage} & \begin{minipage}[b]{\linewidth}\raggedleft
p25
\end{minipage} & \begin{minipage}[b]{\linewidth}\raggedleft
p50
\end{minipage} & \begin{minipage}[b]{\linewidth}\raggedleft
p75
\end{minipage} & \begin{minipage}[b]{\linewidth}\raggedleft
p100
\end{minipage} & \begin{minipage}[b]{\linewidth}\raggedright
hist
\end{minipage} \\
\midrule\noalign{}
\endhead
\bottomrule\noalign{}
\endlastfoot
year & 0 & 1 & 1979.50 & 17.27 & 1952.00 & 1965.75 & 1979.50 & 1993.25 &
2007.0 & ▇▅▅▅▇ \\
lifeExp & 0 & 1 & 59.47 & 12.92 & 23.60 & 48.20 & 60.71 & 70.85 & 82.6 &
▁▆▇▇▇ \\
pop & 0 & 1 & 29601212.32 & 106157896.74 & 60011.00 & 2793664.00 &
7023595.50 & 19585221.75 & 1318683096.0 & ▇▁▁▁▁ \\
gdpPercap & 0 & 1 & 7215.33 & 9857.45 & 241.17 & 1202.06 & 3531.85 &
9325.46 & 113523.1 & ▇▁▁▁▁ \\
\end{longtable}

This is another example of how piping works.

We can also directly use R to call for specific statistics. Say that we
want the mean of a specific variable (say, the mean of Life Expectancy),
we can call the \texttt{mean()} function. There are a few different ways
of referring to variables. We will be using piping once again. In this
case:

\begin{Shaded}
\begin{Highlighting}[]
\NormalTok{gapminder }\SpecialCharTok{|\textgreater{}} 
  \FunctionTok{summarise}\NormalTok{(}\FunctionTok{mean}\NormalTok{(lifeExp))}
\end{Highlighting}
\end{Shaded}

\begin{verbatim}
## # A tibble: 1 x 1
##   `mean(lifeExp)`
##             <dbl>
## 1            59.5
\end{verbatim}

In this case, we pipe the \texttt{gapminder} dataset forward, and use
\texttt{dplyr} to summarize the data, according to the mean. Take a look
at the summarize function in help and look at some of the other options.

\subsection{Plotting}\label{plotting}

We now use the same \texttt{ggplot2} package from earlier. As before,
the \texttt{aes()} section of the code is very important. In this case,
we are providing two ``aesthetics'' options: \texttt{x=continent} and
\texttt{y\ =\ lifeExp}. This code produces two different plots. Based on
just the code, can you guess which two?

\begin{Shaded}
\begin{Highlighting}[]
\FunctionTok{ggplot}\NormalTok{(gapminder, }\FunctionTok{aes}\NormalTok{(}\AttributeTok{x =}\NormalTok{ continent, }\AttributeTok{y =}\NormalTok{ lifeExp, }\AttributeTok{colour =}\NormalTok{ continent)) }\SpecialCharTok{+}
  \FunctionTok{geom\_boxplot}\NormalTok{() }\SpecialCharTok{+}
  \FunctionTok{geom\_point}\NormalTok{()}
\end{Highlighting}
\end{Shaded}

\includegraphics{Lab-0_files/figure-latex/unnamed-chunk-12-1.pdf}

How could we improve this plot?

\begin{Shaded}
\begin{Highlighting}[]
\FunctionTok{ggplot}\NormalTok{(gapminder, }\FunctionTok{aes}\NormalTok{(}\AttributeTok{x =}\NormalTok{ continent, }\AttributeTok{y =}\NormalTok{ lifeExp, }\AttributeTok{colour =}\NormalTok{ continent)) }\SpecialCharTok{+}
  \FunctionTok{geom\_boxplot}\NormalTok{() }\SpecialCharTok{+}
  \FunctionTok{geom\_jitter}\NormalTok{(}\AttributeTok{position =} \FunctionTok{position\_jitter}\NormalTok{(}\AttributeTok{width =} \FloatTok{0.1}\NormalTok{, }\AttributeTok{height =} \DecValTok{0}\NormalTok{), }\AttributeTok{alpha =}\NormalTok{ .}\DecValTok{25}\NormalTok{)}
\end{Highlighting}
\end{Shaded}

\includegraphics{Lab-0_files/figure-latex/unnamed-chunk-13-1.pdf}

Play around with this yourself. Can you make graphs using other
variables? What other types of graphs are there? Feel free to check some
out at:
\url{http://r-statistics.co/Top50-Ggplot2-Visualizations-MasterList-R-Code.html}

\end{document}
